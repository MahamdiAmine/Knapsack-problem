              
%%%%%%%%%%%%%%%%%%%%%%%%%%%%%%%%%%%%%%%%%%%%%%%%%%%%%%%%%%%%%%%%%%%%%
% BY MAHAMDI AMINE
%%%%%%%%%%%%%%%%%%%%%%%%%%%%%%%%%%%%%%%%%%%%%%%%%%%%%%%%%%%%%%%%%%%%%%
\documentclass[12pt]{report}
\usepackage[a4paper]{geometry}
\usepackage[myheadings]{fullpage}
\usepackage{fancyhdr}
\usepackage{lastpage}
\usepackage{graphicx, wrapfig, subcaption, setspace, booktabs}
\usepackage[T1]{fontenc}
\usepackage[font=small, labelfont=bf]{caption}
\usepackage{fourier}
\usepackage[protrusion=true, expansion=true]{microtype}
\usepackage[french,english]{babel}
\usepackage{sectsty}
\usepackage{url, lipsum}
\usepackage[utf8]{inputenc}
\usepackage{indentfirst}
\newcommand{\HRule}[1]{\rule{\linewidth}{#1}}
\onehalfspacing
\setcounter{tocdepth}{5}
\setcounter{secnumdepth}{5}

%-------------------------------------------------------------------------------
% HEADER & FOOTER
%-------------------------------------------------------------------------------
\pagestyle{fancy}
\fancyhf{}
\setlength\headheight{15pt}
\fancyhead[L]{fm\char`_mahamdi@esi.dz    }
\fancyhead[R]{fn\char`_adrao@esi.dz}
\fancyfoot[R]{Page \thepage\ sur \pageref{LastPage}}
%-------------------------------------------------------------------------------
% TITLE PAGE
%-------------------------------------------------------------------------------

\begin{document}
	\renewcommand{\contentsname}{Table des Matières}
	\author{}        
	\date{} 
	\title{  \textsc{ Le problème du Sac à dos}
		\\ [2.0cm]
		\HRule{0.5pt} \\
		\LARGE \textbf{\uppercase{Rapport de Stage }}
		\HRule{2pt} \\ [0.5cm]
		\normalsize \today \vspace*{5\baselineskip}}
	\maketitle
	\tableofcontents
	\renewcommand{\contentsname}
	\newpage
	%------------------------------------------------------------------------------
	% Section title formatting
	\sectionfont{\scshape}
	
	%------------------------------------------------------------------------------
	% introduction
	%------------------------------------------------------------------------------
	\newpage
	\chapter{Introduction}
	%\section{Introduction}
	% \addcontentsline{toc}{section}{Introduction}
	 %\par{}
	 En algorithmique, le problème du sac à dos, noté également KP (en anglais, Knapsack problem) est un problème d'optimisation combinatoire. Il modélise une situation analogue au remplissage d'un sac à dos, ne pouvant supporter plus d'un certain poids, avec tout ou partie d'un ensemble donné d'objets ayant chacun un poids et une valeur. Les objets mis dans le sac à dos doivent maximiser la valeur totale, sans dépasser le poids maximum.
	\section{Approche de solution}
%\addcontentsline{toc}{section}{Introduction}
\par{}
	Pour résoudre ce problème, on peut utiliser la solution classique ou glouton consistant à essayer tous les cas possible. Cela est innéficace pour des valeurs de n supérieur à 8 par exemple.
\par{}
	C'est pour cela que la programmation dynamique est la solution la plus adéquate, et c'est le but de ce TP1.
	
	\section{Principe de la programmtion dynamique}
	La solution récursive est la plus évidente mais la plus coûteuse, la programmation dynamique est par conséquent une amélioration qui consiste à sauvegarder les valeurs des transitions déja calculer pour ne pas répéter les calculs inutile.
	\par{}
	Il existe deux approches:
	\begin{itemize}
		\item \textbf {Approche ascendante}:
		Pour calculer le n ième élément, on commence par calculer le premier élément, puis le deuxième, ... jusqu'à arriver au dernier élément.
		\item \textbf {Approche descendante}:
		Pour calculer le n ième élément, on calcule l'élément n-1 , puis l'élément n-2 , ... jusqu'à arriver au premier élément.
		
	\end{itemize}	
	\chapter{Implémentation de la solution}
	\section{Approche choisie}
	On a utilisé l'approche ascendante (voir introduction)
	\section{Les équations de récurrence}
	(les equations mathématiques à écrire)
	\section{Processus}\begin{enumerate}
		\item On lit la capacité du sac à dos, nommée \emph{maxWeight}.
		\item On lit n objets et pour chaque objet on introduit le poids et le gain correspondant sous forme de tableau comme le montre la figure suivante:
		\item On construit une matrice nommée \emph{matrix} de n+1 lignes (la 1 ère ligne d'indice 0 contient des 0 partout, utile juste pour l'utiliation des équations de réferrence) et maxWeight+1 colone (la 1 ère colone d'indice 0 contient des 0 partout, utile juste pour l'utiliation des équations de réferrence)
		\item On remplit cette matrice en utilisant les équations de réferrence (voir section 2.2).
		\item après la rempli de cette matrice , on trouve le gain maximun dans \emph{matrix[n][maxWeight]}.
	\end{enumerate}
	
		
		
\end{document}
